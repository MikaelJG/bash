
\chapter{Shells}

\section{Bourne Shell}

\textbf{BASH (the Borne Again Shell) and ZSH (the z shell)} are both Unix shells, extentions of the Bourne Shell.

\section{Executable Commands}

Bash uses executables, located in /usr/bin (Really?).

\subsection{ssed}
%keywords:sed

\begin{verbatim}
_sed 's/foo/bar/g' file.txt // Replace all occurrences of "foo" with "bar".
_sed '/pattern/d' file.txt // Remove all lines containing "pattern" from a file.
_sed '/pattern/a new line' file.txt // Add a new line after each line that matches "pattern".
_sed '/0,/foo/s//bar/' file.txt // Replace the first occurrence of "foo" with "bar".
_sed '/\^abc/s/foo/bar/g' file.txt // only in lines that start with "abc", 
// replace all occurrences of "foo" with "bar".
\end{verbatim}

\subsection{Awk}

\begin{verbatim}
awk '{print $1}' file.txt // Print first field (divided space) for each line.
awk '/pattern/' file.txt
awk '2$ ~ / pattern/ {print}' file.txt 
    Print all lines that contain "pattern" in the second field.

awk -F ',' '{sum += 3} END {print sum}' file.csv 
    // Calculate the sum of the values in the third column of a csv.

awk '{sum2+=$2; sum3+=$3} END {print "Sum2: ",sum2, " Avg3: ",sum3/NR}' file.txt
        // Print the sum of values in the second column,
        // and the average of values in the third column of a csv.

awk -F ',' '$1 == $3 {print}' file.csv 
     Print only the lines that have the same value
     in the first and third columns of a csv.
\end{verbatim}

\subsection{Ls}
%keywords:list,allfiles,filesarray,filearray,filear,filesar,listfiles,lsdir,lsdirectory

\begin{verbatim}
all_file=$(ls "${my_path}") // is affected by the shell's
                           // interpretation of characters and whitespaces
                           // use find instead 

all_file=$(find "${my_path}") // is not affected

ls -l | a '{print $5}'// print file size (bytes) for each file in current dir.
\end{verbatim}

\subsection{Wc}
%keywords:linecount,countline,countlines,numlines,numberoflines,numberofline

\begin{verbatim}
NLINES=$(wc -l < "$filename") // file line count
\end{verbatim}

\subsection{Echo}
%keywords:echos

\begin{verbatim}
    echo -e // enables escape sequences.
    echo -e "First line\nSecond line" # \n - new line char.
\end{verbatim}

\subsection{Shopt}
%keywords:shelloptions,shelloption,shellop

\begin{verbatim}
    Set shell options, customize environment and behavior.

    shopt -s // turn on a shell option.
    shopt -u // turn off a shell option.
    shopt // see status of all shell options (without arguments).
\end{verbatim}

\subsection{Cron}
%keywords:job,jobs,jobschedule,schedulejob,schedule,jobsschedule,jobscheduler,scheduler

A time-based job scheduler 
Run tasks or commands at specific intervals (every minute to monthly). 
Useful for backups and system updates 

\begin{verbatim}
edit crontab file (storing commands, times and job recurrence)
\end{verbatim}

\subsection{Synthax}
%keywords:cronex,cronexample

\begin{verbatim}
: MIN HOUR DOM MON DOW CMD

Field    Description    Allowed Value

MIN      Minute field    0 to 59

HOUR     Hour field      0 to 23

DOM      Day of Month    1-31

MON      Month field     1-12

DOW      Day Of Week     0-6

CMD      Command         Any command to be executed.

30 8 * * * /path/to/command.sh // run script everyday 8:30am
\end{verbatim}

\subsection{Ps}
%keywords:process,currentprocess,processes,currentprocesses,cpu,computerprocess,computerprocesses

\begin{verbatim}
no ex yet, show processes
\end{verbatim}

\subsection{Bg}
%keywords:backgroundprocess,backgroundprocesses,background,back

\begin{verbatim}
no ex yet, show background processes
\end{verbatim}

\subsection{Top}
%keywords:topprocess,topprocesses,frontprocess,frontprocesses,front

\begin{verbatim}
top example // continuous
\end{verbatim}

\subsection{Which}
%keywords:wich,wisch,whish

\begin{verbatim}
which gedit // user/bin/gedit 
\end{verbatim}

\subsection{Whereis}
%keywords:where

\begin{verbatim}
no ex yet
\end{verbatim}

\subsection{Dpkg}
%keywords:installedlist,applist,listapp,listapps,listprogram,listprograms,programlist,programslist

List all the installed applications

\begin{verbatim}
dpkg list
dpkg example 
\end{verbatim}

\subsection{Unset}
%keywords:unsetvar,unsetvars

\begin{verbatim}
unset EDITOR, loses default editor variable // removed global
export EDITOR=vim // assigned global
\end{verbatim}

\subsection{Export}
%keywords:setvar,setvars,exportvar,exportvars

\begin{verbatim}
unset EDITOR, loses default editor variable // removed global
export EDITOR=vim // assigned global
\end{verbatim}

\subsection{Bash}

\begin{verbatim}
bash -x complex_script.sh // step-by-step code execution. 
\end{verbatim}

\subsection{Eval}

\begin{verbatim}
Evaluate and execute commands from a string. 
When a string is passed to the eval command and evaluates the resulting command line as if it had been typed on the command line.

eval echo "Hello, $USER!"
eval command will not fail if the command fails.  
\end{verbatim}

\subsection{Uname}

\begin{verbatim}
Print OS basic system information, kernel name, version, release and machine hardware. 
\end{verbatim}

\subsection{Trap}

\begin{verbatim}
no ex yet. Error handling
\end{verbatim}

\subsection{Shred}
\begin{verbatim}
no ex yet.
\end{verbatim}

\subsection{Ln}

\begin{verbatim}
Ln -s // link soft?
\end{verbatim}

\subsection{Curl}

\begin{verbatim}
no ex yet.
\end{verbatim}

\subsection{Wget}

\begin{verbatim}
no ex yet.
\end{verbatim}

\subsection{Head}

\begin{verbatim}
no ex yet.
\end{verbatim}

\subsection{Tail}

\begin{verbatim}
no ex yet.
\end{verbatim}

\subsection{Cmp}

\begin{verbatim}
no ex yet.
\end{verbatim}

\subsection{Diff}

\begin{verbatim}
no ext yet // compare files
\end{verbatim}

\subsection{Sort}

\begin{verbatim}
no ex yet.
\end{verbatim}

\subsection{Grep}

\begin{verbatim}
no ex yet.
\end{verbatim}

\subsection{Strace}

\begin{verbatim}
Debug by tracing a program's system calls and signals.
Its interaction with the OS.

Calls include file operations, network communication,
process management, interrupts or termination signals.
Identify file access problems, permission errors,
resource usage or unexpected behavior. 

Requires root privileges
\end{verbatim}

\chapter{Bash}

\section{bin bash vs bin shell vs bin sh}

\begin{verbatim}

To the OS, the shebang (\#!) specifies the interpreter to use.
To run a bash script, ZSH can invoke the Bash interpreter explicitly.

It will run regardless of the user's default shell.
$ bash script.sh
\end{verbatim}

\section{Dirname (use relative locations)}

\begin{verbatim}
dirname /home/user/Documents/example.txt
# /home/user/Documents

SCRIPT_DIR="$(dirname "${BASH_SOURCE[0]}")"
\end{verbatim}

\section{Globals}
\begin{verbatim}
$HOME - the home directory of the current user.
$PWD - the present working directory.
$OLDPWD - the present working directory.
$USER - the username of the current user.
$SHELL - the name of the current shell.
$PATH - list of directories where executable programs are searched for.
$UID - the user ID of the current user.
$BROWSER - Default browser.
$EDITOR - Default text editor.
$XDG_CONFIG_HOME: $HOME/.config
$LINENO? -
$FUNCNAME[0] - parent function name?
${FUNCNAME[0]}() - Output of the parent function name?
\end{verbatim}

\section{Functions}

\begin{verbatim}
no ex yet.
\end{verbatim}

\subsection{Aliases}

\begin{verbatim}
        .bash_aliases
        alias great='dekg --list | grep "lib"'
\end{verbatim}

\subsection{Installation}

\begin{verbatim}
if ! which "$1"; then
    # if argument given can't be found with which. 
    echo "it is not installed"
fi
\end{verbatim}

\subsection{Uninstall}

\begin{verbatim}
// list all apps with dpkg
sudo apt-get remove <application-name>
// remove any config files associated with app
sudo apt-get purge <application-name>
sudo apt purge?? (recent changes?) 
\end{verbatim}

\subsection{Success}

\begin{verbatim}
if ! which "$x" >/dev/null 2>&1
\end{verbatim}

\subsection{Disregard}

\begin{verbatim}
 >/dev/null // redirect for output discard
ls > output.txt // directs the output to output.txt
ls >/dev/null // auto. discard
\end{verbatim}

\subsection{Conditions}

\begin{verbatim}
if [ -z "$EDITOR" ]; then // if var is unassigned
[ -z "$EDITOR" ] && EDITOR=nano // then assign nano

if [ -f ... ]; then // if its a file
[ -f "$FILE" ] 
[ -d "$FILE" ] // if its a directory
[ -L "$FILE" ] // if its a symbolic link 
[[ -n "$EDITOR" ]] // if its an empty string
[[ -n "$EDITOR" ]] && echo "it is set" || echo "it is not set"

[[ $# -gt 0 ]] // arg. passed is "greater than" 0.
[[ $# -lt 5 ]] // arg. passed is "less than" 5.
[[ "$2" -eq 0 ]] // verify 
\end{verbatim}

\section{Square Brackets}

\begin{verbatim}
[[ and ]] 
    delimits the conditional expression. 
    but they provide more features and are generally preferred.

[ and ]
    test expressions, but have less features.
\end{verbatim}

\section{Special Variables}

\begin{verbatim}
$! // ... 
$? // check the return type
$* // returned output ???
$@ // passed arg. array 
ARGS=("$@") // verify, explain this
ARGS="$@"   // dif?
$# // number of arg. passed.
\end{verbatim}

\section{Logical Operators}

\textbf{Use less if statements}

\begin{verbatim}
& (Simultaniously)
    Execute commands simultaniously, unfinished or not.
&& (if succeeds)
    Only runs the second command if the first one succeeded.
    Command finished.

cat doesnt_exists.txt && echo "this will not be runned" 
[ -z "$EDITOR" ] && EDITOR=nano // not assigned, so
\end{verbatim}

\subsection{Chaining}
Chaining two lines, executes after finished command.

\begin{verbatim}
cat file.txt >> out.txt ; xdg-open out.txt 
\end{verbatim}

\subsection{Double Pipes if fails}

\begin{verbatim}
cat exists.txt || echo "this will not be runned" 
cat doesnt_exists.txt || echo "this will run" 
\end{verbatim}

\section{Comments}

\begin{verbatim}
cat << EOF // or cat <<-EOF
This is a comment:

The EOF token must begin the line,
You can't indent it with its related code. 
EOF

<<-EOF // indentable with Tab only. No white space after EOF.

echo -e "anything\n"
echo -e "will\n"
echo -e "do\n"
\end{verbatim}

\section{Bash Source}

Array of source filename(s) and executed parent script or parent function.
Elements are what executed to reach the current point.
Debug or have context information.

\begin{verbatim}
${BASH_SOURCE[0]}
BASH_SOURCE's 0, script's name or executing function.

function foo {
  echo "Current script : ${BASH_SOURCE[0]}" // ... : /path/to/script.sh
  echo "Parent script is: ${BASH_SOURCE[1]}" // ...: 
}
\end{verbatim}

\section{Refactoring}

\begin{verbatim}
if [[ " $2 " -eq 0 ]];
    then
    $READER_PATH $2 $1
else
    $READER_PATH $1
fi

[[ "$2" -eq 0 ]] && $READER_PATH "$2" "$1" || $READER_PATH "$1"
\end{verbatim}

\section{Command Line}

\subsection{Arguments}

\begin{verbatim}
ARGS=("$@")
\end{verbatim}

\subsection{Set -eu}

\begin{verbatim}
// Shell scripts are cmd lists, runned whether one fails or not.
// It goes on and on. You may not notice one failing. 

set -e // as soon as one command fails, stop script. 
set -u // as soon as you use an undefined variable, 
       // the whole script breaks (stops with error)

set -eu // one liner

grep -eu * // this fails
grep -- -eu * // this succeeds
\end{verbatim}

\subsection{Pipefail}

\begin{verbatim}
// Typically, a piped command is still called even if the first command fails
false | grep -- -eu * // grep is called

set -o pipefail // the whole pipe fails if one cmd fails. 
\end{verbatim}

\subsection{Basedir - solve directories}

\begin{verbatim}
basedir(){
    local script=${BASH_SOURCE[0]}} // this script's location

    if [ -L $script ]; then // if symbolic link
        if readlink -f $script >/dev/null 2>&1; then // resolve link
            script=$(readlink -f $script)
        elif readlink $script >/dev/null 2>&1; then
            script=$(readlink $script)
        elif realpath $script >/dev/null 2>&1; then
            script=$(realpath $script)
        else
            echo "ERROR: Cannot resolve symbolic link $script"
            exit 1 
        fi
    fi

    local dir=$(dirname "$script") // this script's directory
    local full_dir=$(cd "${dir}" && pwd)
    echo ${full_dir} // why do I need {}?
}
\end{verbatim}

\subsection{Flags (hasflag strategy)}

\begin{verbatim}
gh repo create --private --source=. // private + source are flags
hasflag() {
    local flag=${1} // for mandatory arguments. 
    local flag=${1:-} // for optional arguments
                      // convert no arg. to empty string. 
}

hasflag() {
    local flag=${1} // check one given argument (one given flag)
    for arg in $ARGS; do // ARGS("$@")
        if [ "$flag" = "$arg" ]; then // "" since empty flag is possible.
            echo "true"
        fi
    done
    echo "false"
}

if $(hasflag --devopscon); then /genius implementation
    echo "it has --devopscon"
fi

// The echo "true" strategy is genius here. 
// It return the string "true"
// Yet BASH will consider it as a boolean, evaluated in subshell

hasflag() {
    for var in "${ARGS[@]}"; do // check all flags, all args
        for flag in $flags; do // possible flags saved in $flags
            if [ "$var" = "$flag" ]; then
                echo "true"
                return
            fi
        done
    done
    echo "false"
}
\end{verbatim}

\subsection{help}

\begin{verbatim}

    if [ -z "$1" ]; then // --help by default
        ARGS=("--help")
    else
        ARGS=("$@")
    fi

    [ -z "$1" ] && ARGS=("-help") || ARGS=("$@")

    if $(hasflag --help -h); then
        cat << EOT
    
    Usage: kdeploy <subcommand> <opts>
    
    Kubernetes deployment helper
    Commands:
    Options
    -h --help           print help message
    EOT
        exit 0
    fi
\end{verbatim}

\subsection{Options (readopt stategy)}

\begin{verbatim}
// var/${opt}=/ , is a regex tranforming everything after -- in a string. 
// ${opt}*, find all options by finding --

readopt() {
    local opts="$@"
    for var in "${ARGS[@]}"; do
        for opt in $opts; do
            if [[ "$var" = ${opts}* ]]; then
                local value="${var/${opt}=/}"
                if [ "$value" != "$var" ]; then // value could be extracted
                    echo $value
                    return
                fi
            fi
        done
    done
    echo "" // Nothing found
}
\end{verbatim}

\section{Brace expansion}

\begin{verbatim}
echo a{d,c,b}e // Order is prefered, not sorted.
ade ace abe
mkdir /usr/local/src/bash/{old,new,dist,bugs}
chown root /usr/{ucb/{ex,edit},lib/{ex?.?*,how_ex}} // nested
\end{verbatim}

\subsection{Printf}

format and print data like echo, with more formatting options.

\begin{verbatim}
printf format_string [argument1] [argument2] ...
printf "a %s %s beautiful example" very eloquent
printf "Hello, world!\n" // Print a string with a newline character
printf "Pi is approximately %.2f\n" 3.14159 // Pi is approximately 3.14159
\end{verbatim}


